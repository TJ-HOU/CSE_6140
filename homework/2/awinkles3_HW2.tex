\documentclass{article}
\usepackage[utf8]{inputenc}
\usepackage{amsmath}
\usepackage{amsthm}
\usepackage{amsfonts}
\usepackage{amssymb}
\usepackage{amstext}
\usepackage{gensymb}
\usepackage{graphicx}
\usepackage{enumerate}
\pagenumbering{arabic}
\usepackage{fancyhdr}
\usepackage[margin=0.75in]{geometry}
\usepackage{eucal}
\usepackage{parskip} % removes auto indentation for paragraphs
\usepackage{enumitem} % changes the indexing for enumerate
%\setlist[enumerate,1]{label = {(\alph*)}}

\usepackage{mathtools} % used for making a ceiling function

\DeclarePairedDelimiter{\ceil}{\lceil}{\rceil} %create ceiling function as \ceil{x}, \ceil*{x} to add \left and \right

\usepackage{listings} % import code into latex

\usepackage{algorithm} % used to write algorithms
\usepackage[noend]{algpseudocode}

\def\N{\mathbb{N}}
\def\Z{\mathbb{Z}}
\def\Q{\mathbb{Q}}
\def\R{\mathbb{R}}
\newcommand{\Mod}[1]{\ (\text{mod}\ #1)}
\newcommand{\Problem}[1]{\textbf{Problem #1}}
\newcommand{\li}[0]{\liminf_{n\to\infty}}
\newcommand{\ls}[0]{\limsup_{n\to\infty}}
\newcommand{\dl}[2]{\displaystyle\lim_{#1 \to #2}}
\newcommand{\ds}[2]{\displaystyle\sum_{#1}^{#2}}
\newcommand{\ra}{\Rightarrow}

\linespread{1.5}

\usepackage{float}

\title{CSE 6140 - Homework 2}
\author{Alexander Winkles}
\date{}


\begin{document}

\maketitle

\vspace{2in}

For assignment 2, I worked with Willian Kong and Haodong Sun.
I utilized the class textbooks (Kleinberg, CLRS, and Benoit) as well as the website ``GeeksForGeeks'' for different perspectives on designing the algorithms.

\newpage

\Problem{1}

\begin{enumerate}

\item Consider a greedy solution that sorts by smallest $t_i$. 
This is not optimal, consider three emails of weights 1, 2, and 200 with times 1, 2, and 5 respectively. 
The greedy solution gives $\sum_{i=1}^3w_iC_i = 1607$, where if we order by weight this value becomes $1022$. 
Thus this is not optimal by counter example.

\item Consider a greedy solution that sorts by largest $w_i$.
This is not optimal - consider two emails of weights 200 and 100 with times 200 and 1 respectively.
The greedy solution gives 60100, where if we swap the two emails we the value becomes 40300.
Thus this is not optimal by counter example.

\item This greedy algorithm is optimal. 
Consider an optimal solution $O$ and a greedy solution $A$. 
Our goal is to use an exchange argument to gradually transform $O$ into $A$ without increasing the overall cost.
We define an \textit{inversion} to be a pair of jobs $i$ and $j$ such that $\frac{w_i}{t_i} > \frac{w_j}{t_j}$ but $j$ is scheduled before $i$.
By defintion of the greedy algorithm, $A$ has no inversions. 
Because $O$ and $A$ are solutions to the same set of emails, they take the same total amount of time. 
Swapping any inverted jobs will not increase the cost of the solution. 
Now if $O$ has any inversion, then it has a consecutive pair of emails that are inverted and thus can be swapped without increasing the overall cost. 
By swapping all inversions found in $O$, we arrive at a solution identical to $A$.

\end{enumerate}

\Problem{2}

\begin{enumerate}

\item The indices are 4 and 7, with a sum of 32. 

\item 

\textbf{Algorithm:}
\begin{enumerate}

\item Divide the list into two sublists.
\item Return the maximum of the following values:
\begin{enumerate}
\item Maximum sum of left subarray, using a recursive method
\item Maximum sum of right subarray, using a recursive method
\item Maximum sum of subarray that includes middle point
\end{enumerate}

\end{enumerate}

Because the maximum sums of the left and right array use recursive methods, they have a time complexity of $T(n) = 2T(n/2) + \Theta(n)$. 
Using the Master Theorem, this algorithm is $\Theta(n\log{n})$. 

\item 

A linear time algorithm would be such:

\textbf{Algorithm:}

\begin{enumerate}

\item $val1 = val2 = T[0]$
\item for $x \in T - {T[0]}$
\begin{enumerate}
\item $val1 = max{x, val1 + x}$
\item $val2 = max{val1,val2}$
\end{enumerate}
\item return $val2$

\end{enumerate}

\end{enumerate}

\Problem{3}

\begin{enumerate}

\item By the Master Theorem, $a = 49$, $b = 7$, and $d = 2$, so $T(n) \in \Theta(n^2\log{n})$.
\item In this problem $a = \frac{1}{4}$, $b = 9$, and $d = 1$, so the Theorem does not apply as $a \geq 1$ is required. 
\item In this problem $a = 4$, $b = 2$, and $d = 2$. Since this problem is not monotonically increasing, we may not apply the theorem.
\item In this problem $a = 2$, $b = 4$, and $d = 0.6$. Thus by the Master Theorem, $T(n) \in \Theta(n^{0.6})$.
\item In this problem $a = 3$, $b = 2$ and $d = 0$. Thus by the Master Theorem, $T(n) \in \Theta(n^{\log_2{3}})$.

\end{enumerate}

\Problem{4}

\begin{enumerate}

\item To start, we will prove the problem has optimal substructure. 


\end{enumerate}

\end{document}
