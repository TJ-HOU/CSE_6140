\documentclass{article}
\usepackage[utf8]{inputenc}
\usepackage{amsmath}
\usepackage{amsthm}
\usepackage{amsfonts}
\usepackage{amssymb}
\usepackage{amstext}
\usepackage{gensymb}
\usepackage{graphicx}
\usepackage{enumerate}
\pagenumbering{arabic}
\usepackage{fancyhdr}
\usepackage[margin=0.75in]{geometry}
\usepackage{eucal}
\usepackage{parskip} % removes auto indentation for paragraphs
\usepackage{enumitem} % changes the indexing for enumerate
%\setlist[enumerate,1]{label = {(\alph*)}}

\usepackage{mathtools} % used for making a ceiling function

\DeclarePairedDelimiter{\ceil}{\lceil}{\rceil} %create ceiling function as \ceil{x}, \ceil*{x} to add \left and \right

\usepackage{listings} % import code into latex

\usepackage{algorithm} % used to write algorithms
\usepackage[noend]{algpseudocode}

\def\N{\mathbb{N}}
\def\Z{\mathbb{Z}}
\def\Q{\mathbb{Q}}
\def\R{\mathbb{R}}
\newcommand{\Mod}[1]{\ (\text{mod}\ #1)}
\newcommand{\Problem}[1]{\textbf{\large Problem #1}}
\newcommand{\li}[0]{\liminf_{n\to\infty}}
\newcommand{\ls}[0]{\limsup_{n\to\infty}}
\newcommand{\dl}[2]{\displaystyle\lim_{#1 \to #2}}
\newcommand{\ds}[2]{\displaystyle\sum_{#1}^{#2}}
\newcommand{\ra}{\Rightarrow}

\linespread{1.5}

\usepackage{float}

\title{CSE 6140 - Homework 5}
\author{Alexander Winkles}
\date{}


\begin{document}

\maketitle

\vspace{2in}



\newpage

\Problem{1}

Our goal is to minimize the number of gas stations we visit when traveling from Atlanta to New York using a car that can drive at most $K$ miles before needing to refuel. 
Let $D$ be the set of gas stations along the route traveled. 
Then the following algorithm will find the route that minimizes the number of gas stations visited. 

\begin{algorithm}[H]
\begin{algorithmic}[1]
\Procedure{RoadTrip}{$D, K$}
\State $G = K$ \Comment{Assigns the initial amount of gas in the tank}
\State $d_{curr} \gets 0$ \Comment{Assigns the initial gas station stopped at}
\State $T \gets \emptyset$ \Comment{Assigns the initial trip to be the null set}
\While {$D \neq \emptyset$} \Comment{Runs code until car reaches NY}
\State $d_{new} \gets 0$
\State $d_{new} \gets \displaystyle\max_i \{d_i\ |\ (d_i - d_{curr}) < G\}$ \Comment{Finds the further gas station one can travel to from current position}
\If {$d_{new} = 0$} \Comment{This means we cannot advance without filling our tank}
\State $T \gets T \bigcup \{i\}$ \Comment{Adds index from $d_{curr}$ found above to solution}
\State $G = K$ \Comment{Fill up tank}
\Else \Comment{This means we can advance without filling our tank}
\State $G \gets G - (d_{new} - d_{curr})$ \Comment{Finds how much gas is left}
\For {$j \leq i$} \Comment{Removes gas stations from $D$ that are no longer relevant}
\State $D = D - \{d_j\}$
\EndFor
\State $d_{curr} \gets d_{new}$ \Comment{Tank does not need to be filled}
\EndIf
\EndWhile
\State \Return {$T$}
\EndProcedure
\end{algorithmic}
\end{algorithm}

In this algorithm, we find the furthest gas station that our car can travel to and go there.
Once there, we will either travel to the next furthest one possible or have to refuel, which is decided by the if-else statement. 
Note that the car will only travel to a new gas station if it will have some arbitrarily small amount of gas left after arriving, which satisfies the requirement that the car must never run out of gas. 
Analyzing this problem, its time complexity is $\mathcal{O}(n)$, where $n$ is the number of gas stations in $D$, as the worst case scenario is each gas station must be visited. 
(PROVE THIS)

\Problem{2}







\end{document}